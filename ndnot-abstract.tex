% This is "sig-alternate.tex" V2.1 April 2013
% This file should be compiled with V2.5 of "sig-alternate.cls" May 2012
%
% This example file demonstrates the use of the 'sig-alternate.cls'
% V2.5 LaTeX2e document class file. It is for those submitting
% articles to ACM Conference Proceedings WHO DO NOT WISH TO
% STRICTLY ADHERE TO THE SIGS (PUBS-BOARD-ENDORSED) STYLE.
% The 'sig-alternate.cls' file will produce a similar-looking,
% albeit, 'tighter' paper resulting in, invariably, fewer pages.
%
% ----------------------------------------------------------------------------------------------------------------
% This .tex file (and associated .cls V2.5) produces:
%       1) The Permission Statement
%       2) The Conference (location) Info information
%       3) The Copyright Line with ACM data
%       4) NO page numbers
%
% as against the acm_proc_article-sp.cls file which
% DOES NOT produce 1) thru' 3) above.
%
% Using 'sig-alternate.cls' you have control, however, from within
% the source .tex file, over both the CopyrightYear
% (defaulted to 200X) and the ACM Copyright Data
% (defaulted to X-XXXXX-XX-X/XX/XX).
% e.g.
% \CopyrightYear{2007} will cause 2007 to appear in the copyright line.
% \crdata{0-12345-67-8/90/12} will cause 0-12345-67-8/90/12 to appear in the copyright line.
%
% ---------------------------------------------------------------------------------------------------------------
% This .tex source is an example which *does* use
% the .bib file (from which the .bbl file % is produced).
% REMEMBER HOWEVER: After having produced the .bbl file,
% and prior to final submission, you *NEED* to 'insert'
% your .bbl file into your source .tex file so as to provide
% ONE 'self-contained' source file.
%
% ================= IF YOU HAVE QUESTIONS =======================
% Questions regarding the SIGS styles, SIGS policies and
% procedures, Conferences etc. should be sent to
% Adrienne Griscti (griscti@acm.org)
%
% Technical questions _only_ to
% Gerald Murray (murray@hq.acm.org)
% ===============================================================
%
% For tracking purposes - this is V2.0 - May 2012

\documentclass{sig-alternate-05-2015}


\begin{document}

% Copyright
% \setcopyright{acmcopyright}
%\setcopyright{acmlicensed}
%\setcopyright{rightsretained}
%\setcopyright{usgov}
%\setcopyright{usgovmixed}
%\setcopyright{cagov}
%\setcopyright{cagovmixed}


% DOI
% \doi{10.475/123_4}

% ISBN
% \isbn{123-4567-24-567/08/06}

%Conference
% \conferenceinfo{PLDI '13}{June 16--19, 2013, Seattle, WA, USA}

% \acmPrice{\$15.00}

%
% --- Author Metadata here ---
% \conferenceinfo{WOODSTOCK}{'97 El Paso, Texas USA}

%\CopyrightYear{2007} % Allows default copyright year (20XX) to be over-ridden - IF NEED BE.
%\crdata{0-12345-67-8/90/01}  % Allows default copyright data (0-89791-88-6/97/05) to be over-ridden - IF NEED BE.
% --- End of Author Metadata ---

\title{NDNoT: design and implementation of an IoT framework over Named Data Networking}

% \subtitle{[Extended Abstract]

% \titlenote{A full version of this paper is available as
% \textit{Author's Guide to Preparing ACM SIG Proceedings Using
% \LaTeX$2_\epsilon$\ and BibTeX} at
% \texttt{www.acm.org/eaddress.htm}}}

% You need the command \numberofauthors to handle the 'placement
% and alignment' of the authors beneath the title.
%
% For aesthetic reasons, we recommend 'three authors at a time'
% i.e. three 'name/affiliation blocks' be placed beneath the title.
%
% NOTE: You are NOT restricted in how many 'rows' of
% "name/affiliations" may appear. We just ask that you restrict
% the number of 'columns' to three.
%
% Because of the available 'opening page real-estate'
% we ask you to refrain from putting more than six authors
% (two rows with three columns) beneath the article title.
% More than six makes the first-page appear very cluttered indeed.
%
% Use the \alignauthor commands to handle the names
% and affiliations for an 'aesthetic maximum' of six authors.
% Add names, affiliations, addresses for
% the seventh etc. author(s) as the argument for the
% \additionalauthors command.
% These 'additional authors' will be output/set for you
% without further effort on your part as the last section in
% the body of your article BEFORE References or any Appendices.

\numberofauthors{2} %  in this sample file, there are a *total*
% of EIGHT authors. SIX appear on the 'first-page' (for formatting
% reasons) and the remaining two appear in the \additionalauthors section.
%
\author{
% You can go ahead and credit any number of authors here,
% e.g. one 'row of three' or two rows (consisting of one row of three
% and a second row of one, two or three).
%
% The command \alignauthor (no curly braces needed) should
% precede each author name, affiliation/snail-mail address and
% e-mail address. Additionally, tag each line of
% affiliation/address with \affaddr, and tag the
% e-mail address with \email.
%
% 1st. author
\alignauthor
Zhehao Wang\\
       \affaddr{University of California, Los Angeles}\\
       \email{zhehao@cs.ucla.edu}
% 2nd. author
\alignauthor
Jeff Burke\\
       \affaddr{University of California, Los Angeles}\\
       \email{jburke@remap.ucla.edu}
}

\maketitle
\begin{abstract}

\end{abstract}


%
% The code below should be generated by the tool at
% http://dl.acm.org/ccs.cfm
% Please copy and paste the code instead of the example below. 

% \begin{CCSXML}
% <ccs2012>
%  <concept>
%   <concept_id>10010520.10010553.10010562</concept_id>
%   <concept_desc>Computer systems organization~Embedded systems</concept_desc>
%   <concept_significance>500</concept_significance>
%  </concept>
%  <concept>
%   <concept_id>10010520.10010575.10010755</concept_id>
%   <concept_desc>Computer systems organization~Redundancy</concept_desc>
%   <concept_significance>300</concept_significance>
%  </concept>
%  <concept>
%   <concept_id>10010520.10010553.10010554</concept_id>
%   <concept_desc>Computer systems organization~Robotics</concept_desc>
%   <concept_significance>100</concept_significance>
%  </concept>
%  <concept>
%   <concept_id>10003033.10003083.10003095</concept_id>
%   <concept_desc>Networks~Network reliability</concept_desc>
%   <concept_significance>100</concept_significance>
%  </concept>
% </ccs2012>  
% \end{CCSXML}

% \ccsdesc[500]{Computer systems organization~Embedded systems}
% \ccsdesc[300]{Computer systems organization~Redundancy}
% \ccsdesc{Computer systems organization~Robotics}
% \ccsdesc[100]{Networks~Network reliability}


%
% End generated code
%

%
%  Use this command to print the description
%

%\printccsdesc

% We no longer use \terms command
%\terms{Theory}

% \keywords{ACM proceedings; \LaTeX; text tagging}

Named Data Networking of Things (NDNoT) framework is a set of libraries that aim to support application development in a home IoT environment. 
The design of the framework follows the guidelines proposed in ref, specifically providing functionalities that name home devices and their data, discover devices and services, bootstrap devices and define the trust model, and support application level publishing/subscribing. The rest of the abstract introduces Named Data Networking, then describes the design of NDNoT and its considerations for global Internet reachability and constrained devices, and concludes with the description of NDN-Flow, an application developed using this framework.

% Intro
Named Data Networking (NDN) is a proposed future Internet architecture that shifts the network communication model from host-centric to data-centric. Instead of sending packets between source and destination devices identified by IP addresses, NDN disseminates named data at the network layer, and forwards directly using hierarchical and application-meaningful names. Moreover, NDN adopts content-based security and secures data at the time of its generation.

% Naming
In NDNoT, each device, application, and piece of produced data are named. Names having a strong tie with application semantics, the pattern of data retrieval and security model definition makes namespace design a priority. In a home IoT environment, applications usually refer to things and their data using application-meaningful names, which could be different from the names of devices under the home context. For example, /MyHome/devices/wii-controller/2 could name a wii game controller device in my home, which provides user input for a game application \textit{game1} that I run, under the namespace /MyHome/applications/game1/inputs/a. Separating these two namespaces reflects NDN's concept of naming the data the way the application wants, rather than naming the device that produces it. This is helpful in scenarios such as replacing a device with another one, but having them serve the same purpose for the application, in which case application data retrieval won't be hindered because of the replacement.

In NDNoT, we also assume that the devices come with a manufaturer-configured identity, which is another name under the manufacturer namespace, for example, /com/nintendo/wii-controller/serial-1234. This identity corresponds with a certificate, which is another piece of named data that allows the user to verify if the device indeed comes from the manufacturer, before introducing it to the home and naming it accordingly.

To summarize, the namespaces that the framework supports are given in Figure .


% Discovery
Device and service discovery



% Bootstrap and trust
Device bootstrapping and trust model definition

% Application level pub/sub
Application-level publishing and subscribing

% Global reachability
Global reachability

% Constrained devices
Constrained devices

% Flow application
NDN-Flow application

% ref: NDN 2014 CCR
% ref: Named data networking of things, IoTDI 2016
% ref: schematized trust
% github/remap/ndn-flow

\section{}

\end{document}
