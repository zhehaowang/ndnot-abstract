% This is "sig-alternate.tex" V2.1 April 2013
% This file should be compiled with V2.5 of "sig-alternate.cls" May 2012
%
% This example file demonstrates the use of the 'sig-alternate.cls'
% V2.5 LaTeX2e document class file. It is for those submitting
% articles to ACM Conference Proceedings WHO DO NOT WISH TO
% STRICTLY ADHERE TO THE SIGS (PUBS-BOARD-ENDORSED) STYLE.
% The 'sig-alternate.cls' file will produce a similar-looking,
% albeit, 'tighter' paper resulting in, invariably, fewer pages.
%
% ----------------------------------------------------------------------------------------------------------------
% This .tex file (and associated .cls V2.5) produces:
%       1) The Permission Statement
%       2) The Conference (location) Info information
%       3) The Copyright Line with ACM data
%       4) NO page numbers
%
% as against the acm_proc_article-sp.cls file which
% DOES NOT produce 1) thru' 3) above.
%
% Using 'sig-alternate.cls' you have control, however, from within
% the source .tex file, over both the CopyrightYear
% (defaulted to 200X) and the ACM Copyright Data
% (defaulted to X-XXXXX-XX-X/XX/XX).
% e.g.
% \CopyrightYear{2007} will cause 2007 to appear in the copyright line.
% \crdata{0-12345-67-8/90/12} will cause 0-12345-67-8/90/12 to appear in the copyright line.
%
% ---------------------------------------------------------------------------------------------------------------
% This .tex source is an example which *does* use
% the .bib file (from which the .bbl file % is produced).
% REMEMBER HOWEVER: After having produced the .bbl file,
% and prior to final submission, you *NEED* to 'insert'
% your .bbl file into your source .tex file so as to provide
% ONE 'self-contained' source file.
%
% ================= IF YOU HAVE QUESTIONS =======================
% Questions regarding the SIGS styles, SIGS policies and
% procedures, Conferences etc. should be sent to
% Adrienne Griscti (griscti@acm.org)
%
% Technical questions _only_ to
% Gerald Murray (murray@hq.acm.org)
% ===============================================================
%
% For tracking purposes - this is V2.0 - May 2012

\documentclass{sig-alternate-05-2015}


\begin{document}

% Copyright
% \setcopyright{acmcopyright}
%\setcopyright{acmlicensed}
%\setcopyright{rightsretained}
%\setcopyright{usgov}
%\setcopyright{usgovmixed}
%\setcopyright{cagov}
%\setcopyright{cagovmixed}


% DOI
% \doi{10.475/123_4}

% ISBN
% \isbn{123-4567-24-567/08/06}

%Conference
% \conferenceinfo{PLDI '13}{June 16--19, 2013, Seattle, WA, USA}

% \acmPrice{\$15.00}

%
% --- Author Metadata here ---
% \conferenceinfo{WOODSTOCK}{'97 El Paso, Texas USA}

%\CopyrightYear{2007} % Allows default copyright year (20XX) to be over-ridden - IF NEED BE.
%\crdata{0-12345-67-8/90/01}  % Allows default copyright data (0-89791-88-6/97/05) to be over-ridden - IF NEED BE.
% --- End of Author Metadata ---

\title{NDNoT: design and implementation of an IoT framework over Named Data Networking}

% \subtitle{[Extended Abstract]

% \titlenote{A full version of this paper is available as
% \textit{Author's Guide to Preparing ACM SIG Proceedings Using
% \LaTeX$2_\epsilon$\ and BibTeX} at
% \texttt{www.acm.org/eaddress.htm}}}

% You need the command \numberofauthors to handle the 'placement
% and alignment' of the authors beneath the title.
%
% For aesthetic reasons, we recommend 'three authors at a time'
% i.e. three 'name/affiliation blocks' be placed beneath the title.
%
% NOTE: You are NOT restricted in how many 'rows' of
% "name/affiliations" may appear. We just ask that you restrict
% the number of 'columns' to three.
%
% Because of the available 'opening page real-estate'
% we ask you to refrain from putting more than six authors
% (two rows with three columns) beneath the article title.
% More than six makes the first-page appear very cluttered indeed.
%
% Use the \alignauthor commands to handle the names
% and affiliations for an 'aesthetic maximum' of six authors.
% Add names, affiliations, addresses for
% the seventh etc. author(s) as the argument for the
% \additionalauthors command.
% These 'additional authors' will be output/set for you
% without further effort on your part as the last section in
% the body of your article BEFORE References or any Appendices.

\numberofauthors{2} %  in this sample file, there are a *total*
% of EIGHT authors. SIX appear on the 'first-page' (for formatting
% reasons) and the remaining two appear in the \additionalauthors section.
%
\author{
% You can go ahead and credit any number of authors here,
% e.g. one 'row of three' or two rows (consisting of one row of three
% and a second row of one, two or three).
%
% The command \alignauthor (no curly braces needed) should
% precede each author name, affiliation/snail-mail address and
% e-mail address. Additionally, tag each line of
% affiliation/address with \affaddr, and tag the
% e-mail address with \email.
%
% 1st. author
\alignauthor
Zhehao Wang\\
       \affaddr{University of California, Los Angeles}\\
       \email{zhehao@cs.ucla.edu}
% 2nd. author
\alignauthor
Jeff Burke\\
       \affaddr{University of California, Los Angeles}\\
       \email{jburke@remap.ucla.edu}
}

\maketitle
\begin{abstract}

\end{abstract}


%
% The code below should be generated by the tool at
% http://dl.acm.org/ccs.cfm
% Please copy and paste the code instead of the example below. 

% \begin{CCSXML}
% <ccs2012>
%  <concept>
%   <concept_id>10010520.10010553.10010562</concept_id>
%   <concept_desc>Computer systems organization~Embedded systems</concept_desc>
%   <concept_significance>500</concept_significance>
%  </concept>
%  <concept>
%   <concept_id>10010520.10010575.10010755</concept_id>
%   <concept_desc>Computer systems organization~Redundancy</concept_desc>
%   <concept_significance>300</concept_significance>
%  </concept>
%  <concept>
%   <concept_id>10010520.10010553.10010554</concept_id>
%   <concept_desc>Computer systems organization~Robotics</concept_desc>
%   <concept_significance>100</concept_significance>
%  </concept>
%  <concept>
%   <concept_id>10003033.10003083.10003095</concept_id>
%   <concept_desc>Networks~Network reliability</concept_desc>
%   <concept_significance>100</concept_significance>
%  </concept>
% </ccs2012>  
% \end{CCSXML}

% \ccsdesc[500]{Computer systems organization~Embedded systems}
% \ccsdesc[300]{Computer systems organization~Redundancy}
% \ccsdesc{Computer systems organization~Robotics}
% \ccsdesc[100]{Networks~Network reliability}


%
% End generated code
%

%
%  Use this command to print the description
%

%\printccsdesc

% We no longer use \terms command
%\terms{Theory}

% \keywords{ACM proceedings; \LaTeX; text tagging}

Named Data Networking of Things (NDNoT) framework is a set of libraries that aim to support application development in a home IoT environment, with Figure (setup) being an example.

The design of the framework follows the guidelines proposed in (cite:IoTDI15), specifically providing functionalities that name home devices and their data, discover devices and services, bootstrap devices and define the trust model, and support application level publishing/subscribing. The rest of the abstract introduces Named Data Networking, then describes the design of NDNoT and its considerations for global Internet reachability and constrained devices, and concludes with the description of NDN-Flow, an application developed using this framework.

% Intro
Named Data Networking (NDN) is a proposed future Internet architecture that shifts the network communication model from host-centric to data-centric. Instead of sending packets between source and destination devices identified by IP addresses, NDN disseminates named data at the network layer, and forwards directly using hierarchical and application-meaningful names. Moreover, NDN adopts content-based security and secures data at the time of its generation.

% Naming
\textbf{Naming}

In NDNoT, each piece of data, device, and application are named. Namespace design is a priority because names are strongly tied with application semantic, the pattern of data retrieval and trust model definition. In a home IoT environment, applications usually refer to things and their data using application-meaningful names, which could be different from the names of devices under the home context. For example, \textit{/MyHome/devices/wii-controller/2} could name a wii game controller device, which provides user input for a game application \textit{game1} in my home, under the name prefix \textit{/MyHome/applications/game1/inputs/a}. Separating these two namespaces reflects NDN's concept of naming the data the way the application wants, rather than naming the device that produces it. This is helpful in scenarios such as device replacement, where the replaced and replacement serve the same purpose for the application, and application data retrieval won't be hindered because of the replacement.

In NDNoT, we also assume that each device comes with a manufaturer-configured identity, which is another name under the manufacturer prefix, for example, \textit{/Company/Nintendo/wii-controller/serial-1234}. This name serves as a prefix to initialize bootstrapping process for the device, meanwhile it corresponds with a certificate, a piece of named data that allows the user to verify if the device comes from the manufacturer, before introducing it to the home and naming it accordingly.

To summarize, the namespaces are given in Figure (namespace). Suffixes such as ``\_meta'' denotes the metadata related with a device, for example its current status or manufacturer profile. The implementation of NDNoT provides an interface for the user to name their devices, as well as an interface for each device to publish under certain application namespaces.

% Bootstrap and trust
\textbf{Bootstrap and trust model}

Application data authenticity is a primary concern in NDNoT. In NDN, trust decisions can leverage the structure of names to schematize decision-making on a packet-by-packet basis that does not require channel- or session-based semantics {cite: schematized trust}.

To provide authenticity NDNoT follows a hierarchical trust model, in which a home gateway is the trust anchor, and devices should be authorized by the home gateway, or intermediate gateways that are authorized by the home gateway. The authorization process, or device bootstrapping, involves having the gateway sign the certificate of the added device, and installing the certificate of the gateway on the added device. Communications in this process are authenticated by a shared secret, which is manually given to both the gateway and the added device (cite: ndn-pi TR). 

For a consumer application to be able to verify application data it receives, a trust schema, which defines what the expected relationship between data name and the signing certificate name is, is distributed by the gateway. For each type of application data that a device intends to publish, the device asks for permission from the gateway, who keeps a trust schema for each application running in the home environment. Thus upon receiving application data, the consumer would be able to consult the schema and verify it's signed by an expected device, and if that device's certificate can be traced back to the trust anchor.

The trust relationship is summarized in Figure (trust-relationship).

Data confidentiality as a separate concern, could be tackled by name-based access control (cite:nbac-tr). 

% Discovery
\textbf{Device and service discovery}

Device discovery in NDNoT is achieved using synchronization (sync, cite:let's chronosync), which is a multi-party communication paradigm that
efficiently reconciles collections of named data. Each device, upon being added to the home network, will send out an interest under a broadcast devices namespace carrying an initial digest, in order to learn the names of devices in the home from the data response. Afterwards, the new device adds its own name to the set, and announces a new digest so that receiving devices know to fetch the new device's name, and after learning the name fetch the services it provides by appending a ``\_meta'' component to the interest name.

% Application level pub/sub
\textbf{Application-level pub-sub}

Publish-subscribe (pub-sub) is a common communication paradigm for IoT applications. NDN follows a pull-based paradigm: data is delivered based on interests that ask for it, and at network layer persistent subscription with publisher-initiated communication is not supported. NDNoT provides application-layer pub-sub functionality by having the subscriber ``refresh'' its interest for publisher's data. For example, for a publisher whose data is named with incremental sequence numbers, the consumer can keep issueing interest for the next sequence number it expects to receive, and re-issue this interest if it times out.

% Global reachability
\textbf{Global reachability}

While the discussion above focuses on a local context, NDNoT can support the incorporation of the local system with the global Internet in multiple ways. One would be defining a ``globally reachable'' home prefix, such as \textit{/edu/ucla/main-hall/room122}, and name things, devices and their data accordingly. Alternatively, forwarding hints and encapsulation, discussed in (cite:SNAMP), can be utilized to provide global Internet incorporation.

% Constrained devices
\textbf{Constrained devices}

Securing communication for constrained devices arises as a challenge, since they may not have enough computational power or storage to support public key cryptography. To tackle this constraint, NDNoT provides an example which introduces ``helpers'', a more powerful device, for constrained devices. Constained device shares a secret with its helper by encrypting it with the helper's public key, and uses this shared secret to generate HMAC signatures verifiable by the helper. Helper would keep identity names and key pairs for the devices it helps, and packetizes their messages as NDN data packets, and expose them to the network, so that consumers of this data don't need to differentiate constrained devices from others.

% Flow application
\textbf{NDN-Flow application}

NDNoT framework provides implementation in C++, Python, JavaScript and C\#. An application, NDN-Flow, is developed using the framework. NDN-Flow is a home entertainment experience built on top of NDN, in which the player navigates a virtual space by interacting with a webpage on their mobile phones, gyroscopes attached to RFduinos, and have their movement tracked by OpenPTrack (footnote:website), a multi-camera person tracking system running on a PC. In our installation two Raspberry Pis serve as the home gateway, and helper for RFduinos.

% ref: NDN 2014 CCR
% ref: Named data networking of things, IoTDI 2016
% ref: schematized trust
% github/remap/ndn-flow

\end{document}
